% Created 2016-12-14 Wed 00:32
\documentclass[presentation]{beamer}
\usepackage[utf8x]{inputenc}
\usepackage[T1]{fontenc}
\usepackage{fixltx2e}
\usepackage{graphicx}
\usepackage{longtable}
\usepackage{float}
\usepackage{wrapfig}
\usepackage{rotating}
\usepackage[normalem]{ulem}
\usepackage{amsmath}
\usepackage{textcomp}
\usepackage{marvosym}
\usepackage{wasysym}
\usepackage{amssymb}
\usepackage{hyperref}
\tolerance=1000
\usepackage{minted}
\usetheme{metropolis}
\setbeamertemplate{frame footer}{\color{lightgray}Erwin Rooijakkers - Alliander}
\metroset{block=fill}
\usetheme{default}
\author{Erwin Rooijakkers}
\date{14-12-2016}
\title{What is so great about Clojure?}
\hypersetup{
  pdfkeywords={},
  pdfsubject={},
  pdfcreator={Emacs 25.1.1 (Org mode 8.2.10)}}
\begin{document}

\maketitle

\begin{frame}[label=sec-0-1]{Personal journey}
\includegraphics[width=.9\linewidth]{../images/wizard.jpg}
\end{frame}

\section{Clojure is a Lisp}
\label{sec-1}

\begin{frame}[label=sec-1-1]{What is so great about Lisp?}
\begin{quotation}
"The most powerful programming language is Lisp. If you don't know Lisp (or its variant, Scheme), you don't know what it means for a programming language to be powerful and elegant. Once you learn Lisp, you will see what is lacking in most other languages." --- Richard Stallman
\end{quotation}
\end{frame}

\begin{frame}[label=sec-1-2]{Almost no special syntax, only data structures}
\begin{center}
\begin{tabular}{ll}
Lists & \alert{`(1 2 3)}, \alert{`(fred kees piet)}\\
Vectors & \alert{[1 2 3 4 5]}, \alert{[fred kees piet]}\\
Maps & \alert{\{:a 1 :b 2 :b 3\}}, \alert{\{1 kees 2 piet\}}\\
Code & \alert{(+ 1 2 3)}  \alert{;; => 6}\\
Naming & \alert{(def n 10)}\\
Lambda & \alert{(def plus-two (fn [a] (+ a *)))}\\
 & \alert{(plus-two 2)} \alert{;; => 4}\\
Quote & \alert{`(+ 1 2 3)} \alert{;; => (+ 1 2 3)}\\
\end{tabular}
\end{center}
\begin{itemize}
\item Parentheses!
\item Code is data and data is code (homoiconicity)
\end{itemize}
\end{frame}


\begin{frame}[label=sec-1-3]{Extensible}
\begin{itemize}
\item Macro expansion
\end{itemize}
\end{frame}

\begin{frame}[label=sec-1-4]{The Reader parses the Clojure data structures}
\end{frame}
\begin{frame}[fragile,label=sec-1-5]{Phase in compilation that transforms source code: "programs that write programs"}
 \begin{minted}[bgcolor=white,frame=lines]{clojure}
(defmacro unless [pred a b]
  `(if (not ~pred) ~a ~b))

;; usage:
(unless false 
 (println "Will print") 
 (println "Will not print"))
\end{minted}

Not possible to implement as function.
First form is not evaluated!
\end{frame}

\begin{frame}[label=sec-1-6]{Use the command line 1}
\begin{itemize}
\item Get the command line with: Start > Search for 'cmd'
\item Basic commands (replace "<\ldots{}>" by what you want):
\end{itemize}
\end{frame}

\begin{frame}[fragile,label=sec-1-7]{Blocks}
 \begin{block}{A normal block}
\begin{itemize}
\item Use \texttt{@@beamer: arbitrary-command@@} to include an arbitrary \LaTeX{}/Beamer command.
\item Use \texttt{*test*} for \alert{alert}
\item Use \texttt{/italics/} for \emph{italics}
\item Use \texttt{C-c C-b} to change the type of block.
\end{itemize}
\end{block}
\begin{alertblock}{An alert block}
An altert block.
\end{alertblock}
\begin{example}["An example"]

\end{example}
\end{frame}

\begin{frame}[label=sec-1-8]{Learn more:}
\begin{itemize}
\item Rich Hickey - Clojure made Simple: \url{https://youtu.be/VSdnJDO-xdg}
\item Derek Slager - ClojureScript for Skeptics: \url{https://youtu.be/gsffg5xxFQI}
\end{itemize}
\end{frame}
% Emacs 25.1.1 (Org mode 8.2.10)
\end{document}
