% Created 2017-07-03 Mon 11:11
\documentclass[presentation]{beamer}
\usepackage[utf8x]{inputenc}
\usepackage[T1]{fontenc}
\usepackage{fixltx2e}
\usepackage{graphicx}
\usepackage{longtable}
\usepackage{float}
\usepackage{wrapfig}
\usepackage{rotating}
\usepackage[normalem]{ulem}
\usepackage{amsmath}
\usepackage{textcomp}
\usepackage{marvosym}
\usepackage{wasysym}
\usepackage{amssymb}
\usepackage{hyperref}
\tolerance=1000
\usepackage{minted}
\usetheme{metropolis}
\setbeamertemplate{frame footer}{Erwin Rooijakkers}
\metroset{block=fill}
\usetheme{default}
\author{Erwin Rooijakkers}
\date{27-06-2017}
\title{ClojureScript for writing a decentralized app on the Ethereum blockchain}
\hypersetup{
  pdfkeywords={},
  pdfsubject={},
  pdfcreator={Emacs 25.1.1 (Org mode 8.2.10)}}
\begin{document}

\maketitle

\begin{frame}[label=sec-0-1]{Agenda}
\alert{ClojureScript for writing a decentralized app on the Ethereum blockchain}

Blockchain is a decentralised peer-to-peer distributed ledger without the need
for a trusted third party. Bitcoin is a first application of blockchain
technology, limited to the exchange of currency. Ethereum is a later blockchain
with a Turing complete programming language on top in which so-called “Smart
Contracts” can be programmed.

Some expect blockchain technology to cut out the intermediary in various fields
like trading, banking, energy exchange, voting, leasing, and so on. As professor
Egbert-Jan Sol put it on a recent blockchain meetup: “What robots did for the
work force, blockchain will do for the office force.”

Via ClojureScript wrappers around the Ethereum JavaScript API web3.js it’s
possible to create a decentralized app (dapp) in ClojureScript. A great example
of a dapp fully written in ClojureScript is \url{https://ethlance.com/}. Inspired by
this project I am writing a simple website that interacts with a smart contract.

In this presentation I want to tell about blockchain and the way to interact
with it using ClojureScript. Then I will demo a simple application and answer
questions.
\end{frame}

\section{Blockchain}
\label{sec-1}

Video?

\url{https://consensys.net/ethereum/}

\begin{frame}[label=sec-1-1]{Cryptographic Tokens and Addresses:}
\end{frame}
\begin{frame}[label=sec-1-2]{Peer-to-peer Networking}
\end{frame}
\begin{frame}[label=sec-1-3]{Consensus Formation Algorithm}
\end{frame}
\begin{frame}[label=sec-1-4]{Turing Complete Virtual Machine}
\end{frame}
% Emacs 25.1.1 (Org mode 8.2.10)
\end{document}
